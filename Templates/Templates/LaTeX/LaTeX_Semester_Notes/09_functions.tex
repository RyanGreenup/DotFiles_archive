%\documentclass[article]{standalone}
\documentclass[class=article, crop=false]{standalone}

\usepackage{./resources/style}
\usepackage{./resources/referencing}


 \title{(09) Elementary Functions}
\date{Analysis (200023)}
\author{Analysis (200023)}

\begin{document}
%	\maketitle
%	\tableofcontents

\section{Polynomial Functions}
    an $n^{\mathrm{th}}$ degree polynomial is given by:

    \[
      p\left( z \right)  =  w =  \alpha_0 +  \alpha_1z_1+  z_2 \dots \alpha_nz_n
    \]
        where:
    \begin{itemize}
      \item $\alpha$ is a complex constant
      \item $z$ is a complex variable
    \end{itemize}

    \section{Rational Function}
    A rational function is composed of two polynomial functions $P\left( z \right) $ and $Q\left( z \right) \neq 0$ :
    \[
      \frac{P\left( z \right) }{Q\left( z \right) }
    \]


\newpage


    \section{Euler's Formula}
    This is really important, and there are lots of proofs for it, for the most part though, just come to accept it, the best justification for \textit{Euler's Formula} that I've seen is by \href{https://www.youtube.com/watch?v=mvmuCPvRoWQ}{\textit{3Blue1Brown}}, Seriously, watch this, it's really good.

    \begin{align*}
      e^{i\theta} &=  \cos{\theta} +  i \cdot  \sin{\theta} \\
      \implies   r \cdot e^{i\theta} &= r \left( \cos{\theta} +  i \cdot  \sin{\theta} \right) \\
      r \cdot  e^{i\theta} &=  r \cdot  \mathrm{cis}{\left( \theta \right) } \\
      &= z
    \end{align*}

    It is not uncommon to see Euler's formula used instead of polar notation, particularly in the \textit{Churchill} text, don't stress out, it has the same meaning in effect.

    \section{Exponential Function}
    The exponential function is defined for all complex values:
    \begin{align*}
      e^z =  e^{x +  iy} &= e^x \cdot e^iy \\
      &= e^x \cdot  \left(  \cos{y} +  i \cdot  \cos{y} \right)
    \end{align*}
    \subsection{Modulus and Argument}
    because:
    \[
    e^z = e^x\cdot \mathrm{cis}{\left( y \right) }
    \]
\noindent    we have the modulus:
    \[
            \left| e^z \right| = e^x
    \]

\noindent and the argument:
    \[
        \operatorname{arg}\left( e^z \right) = y + 2\pi k \qquad \left( k \in \mathbb{Z}^*  \right)
    \]\\

    \hfill\begin{minipage}{\dimexpr\textwidth-3cm}
      the notation $\mathbb{Z}^* $ means 'non-negative integers' so, $\left( 0, 1, 2, 3 \dots \right) $
    \end{minipage}
    \ \




    \section{Properties of the exponential function}
    \paragraph{Periodicity}
    The complex exponential function is periodic:
    \[
      e^{z +  2\pi i} =  e^z
    \]
    This corresponds the sliding action of $z +  2 \pi i$ to a rotation of $x +  2 \pi$ radians of rotation, because the rotation has a period of $2 \pi$, the rotation will be $x$ radians.

    For this reason we also have:
    \begin{align*}
    e^z &=  e^z\\
    \left( e^z \right)^{\frac{1}{n}}  &=  e^{\frac{z +  2 k \pi}{n}}
    \end{align*}

    \paragraph{Principle Argument}
    We choose the principle argument as $(-\pi, \pi]$, this is useful later on for dealing with logs.
    \begin{itemize}
      \item The Principle Argument of a complex function is denoted by a capital:
        \subitem $\operatorname{Arg}\left( r \cdot \mathrm{cis}{\theta} \right) = \theta$
      \item The general Argument of a complex function is denoted by lower case:
        \subitem $\operatorname{arg}\left( r\cdot \mathrm{cis}{\left( \theta \right) } \right) = \theta +  2\pi k \qquad :\enspace k \in \mathbb{Z}^+$
    \end{itemize}

    \paragraph{Differentiating $e^z$}
    So if you want to just move on:
    \[
    \frac{\operatorname{d} }{\operatorname{d} z}\left[ e^z \right] = e^z
    \]
    If you want to know why:

    \begin{align*}
      \frac{\operatorname{d} }{\operatorname{d} z}\left( e^z \right) &= \frac{\partial }{\partial x} \left[ e^{x +  iy} \right] +  i \cdot \frac{\partial }{\partial y}\left[ e^{x + iy} \right] \\
      &= \frac{\partial }{\partial x}\left[ e^x \cos{y} \right] + \frac{\partial }{\partial x}\left[ e^x \sin{y} \right] \\
      &= \cos{y}e^x +  \sin{y}e^x\\
      &= e^{x + iy}\\
      &= e^z
    \end{align*}
    \begin{flushright}
    {\rule{0.7em}{0.7em}}
    \end{flushright}

    \paragraph{Polar Form}
    The exponential function can be expressed as:
    \begin{align*}
    e^z &= e^x \cdot \left( \cos{y} +  i \sin{y} \right) \\
    &= e^x\cdot \mathrm{cis}{y}
    \end{align*}


    \begin{alignat*}{5}
    & \implies  &\operatorname{Arg}\left( e^z \right)& &= &y\\
    & \implies  &\operatorname{arg}\left( e^z \right)& &= &y + 2\pi\\
    &  \implies      &\left| e^z \right|& &= &e^x
    \end{alignat*}

    \paragraph{Additional Properties}
    Consider that $\forall x,y \in \mathbb{R} $ $e^x \neq 0$ and also that $e^{iy} \neq 0$, hence:
    \[
      e^{z} = e^x \cdot e^{iy} \neq 0
    \]

    Although many things carry over, be careful because:
    \[
      e^x \nleq 0 \qquad \text{   however   } \qquad e^z < 0 \quad \vee  \quad  e^z > 0
    \]

\newpage

    \section{The Log Function}


    \subsection{Deriving the Log Function}
    With real variables the motivation for the log function is a function for $y$ such that:
    \[
    e^y = x
    \]
    In the complex case the motivation is much the same:
    \[
    e^w =  z
    \]
    So in order to solve a value for the complex logarithm express $z$ in polar form and decompose $w$ into a real and complex part:

   \begin{align*}
   e^w =  z  \implies   e^{u +  iv} &=  r e^{i\theta}\\
   	e^u \cdot  e^{iv} &=  re^{i\theta}
  	\end{align*}

        Equate the real and imaginary parts

\begin{multicols}{2}
  \begin{align*}
    e^u &= r\\
     \implies  u &= \ln{\left( r \right) }
  \end{align*}\break
  \begin{align*}
      e^{iv} &= e^{i\theta}\\
      \implies  v &=  \theta +  2\pi k, \qquad k \in \mathbb{Z}^+
  \end{align*}
\end{multicols}

so the equation $e^w= z$ is satisfied if and only if:
\begin{align*}
  w &=   u +  iv\\
  &=  \ln{\left( r \right) } +  i \cdot  \left(  \theta +  2\pi k \right) , \qquad k \in \mathbb{Z}^+\\
    &= \ln{\left|      z   \right| } +  i \cdot \operatorname{arg}\left( z \right)
\end{align*}


\ \

\ \

\hfill\begin{minipage}{\dimexpr\textwidth-3cm}
\begin{tcolorbox}

  \subparagraph{Complex Logarithm}
  A value in the complex plane $z= r\cdot e^{i\cdot \theta}=  r\cdot \mathrm{cis}{\left( \theta \right) } \in \mathbb{C} $ will have a logarithm:
\begin{align}
  \log_e{\left( z \right) } = \ln{     \left| z \right| + i\cdot \operatorname{arg}\left( z \right)  } \label{compdef}
\end{align}
\end{tcolorbox}

\end{minipage}
\ \\

The Complex Logarithm isn't a function, it is what is known in complex analysis as a '\textit{multiple-valued function}' which is merely a \href{https://en.wikipedia.org/wiki/Multivalued_function}{\textit{binary relation}} \footnote{\url{https://en.wikipedia.org/wiki/Multivalued_function}}, this is really ambiguous though because clearly a function can only have one output, the terminology is used to respect the fact that if, for example, $\log_e{\left( z \right) }$, is restricted to a single branch, it is indeed a function.
\ \\



\newpage



\subsection{Notation}



    \paragraph{Base}\ \\
    Also another really confusing point is that often times in complex analysis $\operatorname{log}\left( z \right)$ is used to represent the complex natural logarithm and $\ln{\left( x \right) }$ is used to represent the real natural logarithm, so you might see $\operatorname{log}$ and immediately think base-10, but don't, we deal excusively in $e$ with complex awnalysis.\\

    It's all redundant anyway because the change of base formula applies also to complex logarithm's anyway.\\

    For clarity sake, I'll write $\log_e{\left(  \right) }$ when dealing with complex natural logarithms and $\ln{\left(  \right) }$ when dealing with real natural logarithms.

    \subsection{Change of Base}
    So in this example our desired solution is $y$ :

    \begin{align*}
      10^y &= e +  4i\\
      \mathrm{Log}_e{\left( 10^y \right) } &=  \mathrm{Log}_e{\left( 3 +  4i \right) } \\
      y \cdot  \mathrm{Log}_e{\left( 10 \right) }&= \mathrm{Log}_e{\left( 3 +  4i \right) }\\
      y &=  \frac{\mathrm{Log}_e{\left( 3 +  4i \right) }}{\mathrm{Log}_e{\left( 10 \right) }} \\
&\implies      \mathrm{Log}_{10}{\left( 3 + 4i \right) } =  \frac{\mathrm{Log}_e{\left( 3 +  4i \right) }}{\mathrm{Log}_e{\left( 10 \right) }}
    \end{align*}
    \begin{flushright}
    {\rule{0.7em}{0.7em}}
    \end{flushright}


    This works because $\mathrm{Log}_e{\left(  \right) }$ is a valid function, it should work with other branches but I'm not sure.


\newpage



\paragraph{Arguments}\ \\
%The exponential function is periodic, so, it stands to reason that there will be multiple solutions for the logarithmic function.\\
The principal argument corresponds to $\theta \in (-\pi,\pi]$ and is distinguished by using uppercase:
    \begin{itemize}
      \item Principal Argument:
\subitem        $\operatorname{Arg}\left( z \right) = \Theta \qquad -\pi < \Theta \leq \pi$
\subsubitem So less that or equal to $\pi$ but $\Theta \neq \pi$
\item Argument:
  \subitem $\operatorname{arg}\left( z \right) = \theta =  \Theta + 2\pi k  \qquad \left( k \in \mathbb{Z}  \right) $
  \subsubitem This is periodic.
    \end{itemize}
    So for example, we may have $r \cdot \mathrm{cis}{\left( \theta \right) } =  r\cdot \mathrm{cis}{\left( \Theta \right) }$ the difference being that
    \begin{itemize}
      \item $\theta \in \mathbb{R} $
      \item $\Theta \in (-\pi, \pi]$
    \end{itemize}

    \paragraph{Logarithms}
    \ \

    \hfill\begin{minipage}{\dimexpr\textwidth-3cm}
    \begin{tcolorbox}

    \paragraph{Notation}\ \\
    Where $z =  r \cdot \mathrm{cis}{\left( \theta \right) }$:\\
    \ \\
    The \textbf{Principal Value} of the $\log_e{\left(  \right) }$ function is given by:
    \begin{align}
      \mathrm{Log}_e{\left( z \right) } &= \ln{     \left| z \right|  } +  i \cdot  \operatorname{Arg}\left( z \right) \\
      &= \ln{ \left( r \right)  } +  i \cdot \Theta
      \label{prinlogdef}
    \end{align}

    The \textbf{multi-valued} log function is given by:

    \begin{align}
      \log_e{\left( z \right) } &= \ln{     \left| z \right|  } +  i \cdot \operatorname{arg}\left( z \right) \\
      &= \ln{ \left( r \right)  } +  i \cdot \left( \Theta + 2\pi k \right) \qquad k \in \mathbb{Z}
      \label{mvlogfunc}
    \end{align}
   \end{tcolorbox}

    \end{minipage}
    \ \\


    \newpage



      \subsection{Branches}
      A branch of a function $f\left( z \right) $ that has multiple outputs  (e.g. $\log_e{\left( z \right) }$), is a function with a single output that is analytic in the domain. \\
At every point of the domain, the single-valued function must assume exactly one of the various possible values that the original function might have given as output.\\
\ \


\hfill\begin{minipage}{\dimexpr\textwidth-3cm}
  The requirement of analyticity prevents $F\left( z \right) $ from taking on a random selection of the values of $f\left( z \right) $.\\
  So in the case of logs:\\
\end{minipage}

\ \

\hfill\begin{minipage}{\dimexpr\textwidth-3cm}
\begin{tcolorbox}

  \subparagraph{Logarithmic Branches}\ \\

  The \textbf{Principal Branch} of the logarithmic function is given by:
  \begin{align}
   \mathrm{Log}_e{\left( z \right) } &= \ln{ \left( r \right)  } +  i \cdot \Phi \qquad -\pi < \Phi < \pi
    \label{pblog}
  \end{align}
  Any \textbf{Branch} of the logarithmic function is given by:
  \begin{align}
    \log_e{\left( z \right) } &= \ln{ \left( r \right)  } +  i \cdot \phi \qquad -\alpha < \phi <\alpha + 2 \pi k
    \label{branchlogdef}
  \end{align}
\end{tcolorbox}

\end{minipage}
\ \\

 here, the \textbf{Principal Value} of the complex log and the \textbf{Principal Branch} are denoted ambiguously by $\mathrm{Log}_e{\left(  \right) }$ but they are both different, for example $\mathrm{Log}_e{\left( -10 \right) }= \ln{ 10 } +  \pi\cdot i$ as the principal value, however, it is entirely undefined on the principal branch.\\

 So basically, on the principal branch, we delete the entire non-positive $x$-axis (i.e. including zero), where as the principal value is allowed to take values there using $\pi$ (but not $-\pi$).\\

 Now one might ask 'why on Earth would we do such a confusing and ambiguous thing?'\\
 The reason is because it is the only way to make the $\log_e{\left(  \right) }$ function continuous and hence analytic.:\\
 \ \

 \hfill\begin{minipage}{\dimexpr\textwidth-3cm}
   Clearly we couldn't use the \textit{multi-valued} log functions as in (\ref{mvlogfunc}) because that has multiple outputs for a given input and so it is not a function, hence it clearly is not analytic.\\

   Now why can't we restrict the domain to the principal argument of $\theta \in (-\pi, \pi]$?\\

   \ \

   \hfill\begin{minipage}{\dimexpr\textwidth-3cm}
   A function must be continuous in order to be differentiable and hence analytic, the condition for continuity is:\\
   \[
   \lim_{z     \rightarrow \alpha}\left[ \mathrm{Log}_e{\left( z \right) } \right] = \mathrm{Log}_e{\left( \alpha \right) }
   \]
But if we take a value on the negative $x$-axis:

\begin{align*}
    \lim_{z     \rightarrow -1}\left[ \mathrm{Log}_e{\left( z \right) } \right] &= + \pi \cdot i  \qquad \left( \text{From Above} \right) \\
    \lim_{z     \rightarrow -1}\left[ \mathrm{Log}_e{\left( z \right) } \right] &= -  \pi \cdot i  \qquad \left( \text{From Below} \right)
\end{align*}

This implies that the limit does not exist, similarly it can be shown that the limit does not exist as $z$ approaches values on the non-posiitive $x$-axis, no limit means no continuity, which  means no differentiablility, which means not analytic.\\

The solution here is to remove the negative $x$-axis and zero from the domain, then the $\mathrm{Log}_e{\left(  \right) }$ function will be analytic.j
\ \\





   \end{minipage}
   \ \


 \end{minipage}
 \ \



\newpage



\paragraph{Branch Cuts}
A branch cut is a curve that is introduced in order to define the multiply defined function $F\left( z \right) $, it's basically the line we have to delete.\\

So imagine $\log_e{\left( z \right) }$:
\ \

\ \
% \begin{figure}[h!]
%
% 	\includegraphics[width=0.4\linewidth]{}
% 	\caption{Polar Representation of $z\in \mathbb{C}$}
% 	\label{fig:handwriting}
% \end{figure}

The principal branch deletes the negative $x$-axis and $0$ as shown in blue below:

% \begin{figure}[h!]
% 	\includegraphics[width=0.4\linewidth]{"handwriting2"}
% 	\caption{Principal Branch Cut}
% 	\label{fig:handwriting-2}
% \end{figure}

\newpage


So for the Complex Logarithm that we used as a definition at (\ref{branchlogdef}), the angle $\alpha$ from the origin is used as the branch cut in order to define a branch.

\ \

\hfill\begin{minipage}{\dimexpr\textwidth-3cm}
  So for example, as above in Figure \ref{fig:handwriting-2}, the line drawn from $0$ at $-\pi$ radians corresponds to $\alpha = - \pi$ in the definition; This line is a branch cut of $\log_e{\left( z \right) }$ that we use to define the principal branch of the logarithmic function:
  \[
  \log_e{\left( z \right) } = \ln{ \left( r \right)  }+ i\cdot \Phi \qquad -\pi < \Phi < \pi
  \]
\end{minipage}
\ \\

Basically you just delete that line in the definition of the domain.\\

Also be aware that the branch cut doesn't have to be a line, it could be a curve, this diagram is an example of a branch cut that would define a valid branch of the $\log_e{\left( z \right) }$ function:

% \begin{figure}[h!]
% 	\includegraphics[width=0.4\linewidth]{"handwriting3"}
% 	\caption{}
% 	\label{fig:handwriting-3}
% \end{figure}

It works because however far the value of $z$ is from the origin, the values of $\theta$ will still be restricted to one revolution.

\subsection{Principal Branch as opposed to Principal Value}
Be aware that there is a difference between the \textit{principal value} and the \textit{principal branch} of the complex natural logarithm:
\begin{itemize}
  \item The \textbf{Principal Value} of the complex logarithmic function of $z = r\cdot \mathrm{cis}{\left( \theta \right) }$ is:
    \subitem $\log_e{\left( z \right) } = \ln{ \left( r \right)  }+ i \cdot \Theta \qquad \Theta \in  (-\pi, \pi]$
  \item The \textbf{Principal Branch} of the complex logarithmic function of $z = r\cdot \mathrm{cis}{\left( \theta \right) }$ is:
    \subitem $\log_e{\left( z \right) } =  \ln{ \left( r \right)  +  i \cdot \Phi \qquad -\pi < \Phi < \pi }$
\end{itemize}

So to be clear, the principal value of $\mathrm{Log}_e{\left( -1 \right) } = \pi i$, however this is entirely undefined on the principal branch.



\newpage

    \subsection{Properties of Complex Logs}
    Familiar properties of logartithms in calculus are sometimes but not aways true in complex analysis, the problem is usually the multiple branches of the log function.

    So one of the first things to be careful with is:
    \[
      e^{\log_e{\left( z \right) }} = z \qquad \text{by definition}
    \]
    However:

    \[
    \log_e{\left( e^z \right) } =  x +  i \cdot \left( y +  2\pi k \right)  \qquad k \in \mathbb{Z}^+ \\
    \]
    But if we take the principal branch:
      \begin{align*}
        \mathrm{Log}_e{\left( e^z \right) } &=  x +  i \cdot  y  \\
        &= e^z
      \end{align*}

      Another one to be careful of, for example:
      \[
        \log_e{\left( i^2 \right) }\neq \frac{1}{2} \cdot \log_e{\left( i \right) }
      \]
      because:
      \begin{multicols}{2}
        \begin{align*}
         \log_e{\left( i^2 \right) } =  \log_e{\left(  - 1 \right) } &= \ln{      \left|  - 1 \right|  } +  i \cdot  \operatorname{arg}\left( - 1 \right) \\
         &=  0 +  \operatorname{arg}\left( - 1 \right) \cdot i\\
         &=  \operatorname{arg}\left( - 1 \right) \cdot i\\
         &= \left( \pi +  2 \pi k \right) \cdot  i\\
         &=  \pi \left(1 +  2k \right) \cdot i
        \end{align*}\break
        \begin{align*}
          2\cdot \log_e{\left( i \right) }&= 2\left[ \ln{     \left| i \right|  }+ \operatorname{arg}\left( i \right) \cdot i \right] \\
          &= 2\left[ o +  \left( \frac{\pi}{2} +  2 \pi k \right) \cdot i \right] \\
          &= i \cdot  \left[  \pi +  4 \pi k \right] \\
          &= \pi\cdot i \left[ 4k+ 1 \right]
        \end{align*}
      \end{multicols}

      If however we choose the principal branch that corresponds to $k =  0$:
      \[
      \log_e{\left( i^2 \right) } = \pi i =  2 \cdot \log_e{\left( i \right) }
      \]



      \newpage

\subsection{Log Laws}
\paragraph{Addition/Multiplication}\ \\
because:
\begin{align*}
\operatorname{arg}\left( z_1 \cdot z_2 \right) &= \operatorname{arg} \left( z_1 \right) +  \operatorname{arg}\left( z_2 \right) \\
\ln{ \left( x_1 \cdot  x_2 \right)  } &=  \ln{ \left( x_1 \right)  }+  \ln{  \left( x_2 \right)  }
\end{align*}
We have:
\ \

\hfill\begin{minipage}{\dimexpr\textwidth-3cm}
\begin{tcolorbox}

Addition of Logs
\begin{align}
  \log_e{\left( z_1 \cdot z_2 \right) }&= \log_e{\left( z_1 \right) } +  \log_e{\left( z_2 \right) }\\
  \log_e{\left( \frac{z_1}{z_2} \right) } &= \log_e{\left( z_1 \right) } -  \log_e{\left( z_2 \right) } \qquad \left( z_2 \neq 0 \right)
\end{align}
\end{tcolorbox}

\end{minipage}
\ \\

Also because:
\[
    \left| z_1 \cdot  z_2 \right|  =      \left| z_1 \right|  \cdot     \left| z_2 \right|
\]
we have:
\[
\log_e{\left(     \left| z_1 \cdot z_2 \right|  \right) } =  \log_e{\left(     \left| z_1 \right|  \right) } +  \log_e{\left(     \left| z_2 \right|  \right) }
\]


\subparagraph{Relationships to Roots}
\begin{align*}
    z^n &=  e^{n\cdot \log_e{\left( z \right) }}\\
    z^\frac{1}{n} &=  e^{\frac{1}{n}\cdot \log_e{\left( z \right) }}\\
    &= e^{\frac{1}{n} \ln{ r } +  \frac{i\left( \theta 2\pi k \right) }{n}}\\
    &= \sqrt[n]{r} \cdot  e^{i \left( \frac{\theta}{n} +  \frac{2k \pi}{n} \right) }\\
      &= \sqrt[n]{r} \cdot \mathrm{cis}{\left( \frac{\theta}{n} +  \frac{2\pi k}{n} \right) }
\end{align*}

\paragraph{Exponentiation}
\begin{align*}
  \log_e{\left( z^n \right) } &=  \ln{      \left| z^n \right|  } +  \operatorname{arg}\left( z^n \right) \\
  &= \ln{ \left(     \left| z \right| ^n \right)  } +  \operatorname{arg}\left( \left[ r\cdot \mathrm{cis}{\theta}^n \right]  \right) \\
  &= n\cdot \ln{     \left| z \right|  } +  \operatorname{arg} \left( rn \cdot \mathrm{cis}{\left( n\theta \right) } \right) \\
  &= n\cdot \ln{     \left| z \right|  } +  \operatorname{arg}\left( \mathrm{cis}{\left( \theta\cdot n \right) } \right) \\
&=   n\cdot \ln{     \left| z \right|  } +  \left( n\theta +  2\pi k \right) \qquad \exists k \in \mathbb{Z}\\
\intertext{This is only valid for a branch of the logarithmic function, in this case choose the principal branch}
  &= n\cdot \ln{     \left| z \right|  }+ n\theta\\
  &= n\cdot \mathrm{Log}_e{\left( z^n \right) }
\end{align*}

Therefore the exponential log law carries over where we use a single branch of the complex logarithmic function.


\newpage

\subsection{Differentiating $\log_e{\left( z \right) }$}

\hfill\begin{minipage}{\dimexpr\textwidth-3cm}
\begin{tcolorbox}
  \paragraph{Derivative of complex log}
\begin{align}
  \frac{\operatorname{d} }{\operatorname{d} z}\left( \mathrm{Log}_e{\left( z \right) } \right) &= \frac{1}{    \left| z \right| \cdot \mathrm{cis}{\left( \operatorname{Arg}\left( z \right)  \right) }} \label{logdiff}
\end{align}
If you're not dealing with the principal branch adjust the argument accordingly, the derivative is only defined for a branch of the logarithmic function.
\end{tcolorbox}

\end{minipage}
\ \\
\ \\
The working for this is pretty straight-forward.
\begin{align*}
  f\left( z \right) = \log_e{\left( z \right) } &= \ln{     \left| z \right|  } +  i \left( \operatorname{arg}\left( z \right) + 2\pi k \right) \\
  \intertext{Let $\theta =  \operatorname{arg}\left( z \right) $}
  &= \ln{     \left| z \right|  } + \left( \theta +  2\pi k \right) \\
  \intertext{Choose a branch of $\log_e{\left( z \right) }$ so that it is a function}
  &= \ln{     \left| z \right| + i\Phi}  \qquad -\alpha < \Phi < \alpha + 2\pi k
  \intertext{let $r =     \left| z \right| $}
  &= \ln{ \left( r \right)  } +  i \Phi\\
\end{align*}
\begin{multicols}{2}
  Let $u =  \ln{ \left( r \right)  }$:
  \begin{flalign*}
      u_r &= \frac{1}{r}&\\
      u_\Phi &= 0
  \end{flalign*}\break
  Let $v =  \Phi $:
  \begin{flalign*}
    v_\Phi &= 1&\\
    v_r &= 0
  \end{flalign*}
\end{multicols}
Our domain is $r>0$, so $\frac{\partial u }{\partial r}$ and $\frac{\partial v }{\partial \Phi}$ are continuous on the entire domain.\\

because $f\left( z \right) =  u\left( r,\Phi\right)  +  i \cdot  v\left( r, \Phi \right) $, use the polar form of the \textit{Cauchy Riemann} equations :
\ \\
\begin{multicols}{2}
  First Condition:
  \begin{flalign*}
    r\cdot u_r &= v_\Phi&\\
    r \cdot \frac{\partial }{\partial r} \cdot \left( \ln{ \left( r \right)  } \right) &= \frac{\partial }{\partial \Phi}\left( \Phi \right) &\\
    r \times \frac{1}{r}&= 1 &\\
    1 &= 1
  \end{flalign*}\break
  Second Condition:
  \begin{flalign*}
      u_\Phi &=  - r\cdot v_r&\\
      \frac{\partial }{\partial \Phi}\left( \ln{ \left( r \right)  } \right) &= - r \times \frac{\partial }{\partial r}\left( \Phi \right) &\\
      0 &=  - r \times 0 &\\
      0 &= 0
  \end{flalign*}
\end{multicols}
both the \textit{Cauchy Riemann} equations are satisfied so:
\begin{align*}
    \frac{\operatorname{d} }{\operatorname{d} z}\left( \log_e{\left( z \right) } \right) = e^{- i\Phi} \cdot \left( u_r + iv_r \right) \\
    &= e^{- i\Phi}\cdot \left( \frac{1}{r} + i \times 0 \right) \\
    &= \frac{1}{re^{i\Phi}}\\
    &= \frac{1}{r\cdot \mathrm{cis}{\left( \Phi \right) }} \\
    &= \frac{1}{    \left| z \right| \cdot \mathrm{cis}{\left( Arg\left( z \right)  \right) }}
\end{align*}
 \ \




\end{document}
