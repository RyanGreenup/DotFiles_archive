

\iffalse
\documentclass[]{article}
\usepackage{lmodern}
\usepackage{amssymb,amsmath}
\usepackage{ifxetex,ifluatex}
\usepackage{fixltx2e} % provides \textsubscript
\ifnum 0\ifxetex 1\fi\ifluatex 1\fi=0 % if pdftex
  \usepackage[T1]{fontenc}
  \usepackage[utf8]{inputenc}
\else % if luatex or xelatex
  \ifxetex
    \usepackage{mathspec}
  \else
    \usepackage{fontspec}
  \fi
  \defaultfontfeatures{Ligatures=TeX,Scale=MatchLowercase}
\fi
% use upquote if available, for straight quotes in verbatim environments
\IfFileExists{upquote.sty}{\usepackage{upquote}}{}
% use microtype if available
\IfFileExists{microtype.sty}{%
\usepackage[]{microtype}
\UseMicrotypeSet[protrusion]{basicmath} % disable protrusion for tt fonts
}{}
\PassOptionsToPackage{hyphens}{url} % url is loaded by hyperref
\usepackage[unicode=true]{hyperref}
\hypersetup{
            pdfborder={0 0 0},
            breaklinks=true}
\urlstyle{same}  % don't use monospace font for urls
\IfFileExists{parskip.sty}{%
\usepackage{parskip}
}{% else
\setlength{\parindent}{0pt}
\setlength{\parskip}{6pt plus 2pt minus 1pt}
}
\setlength{\emergencystretch}{3em}  % prevent overfull lines
\providecommand{\tightlist}{%
  \setlength{\itemsep}{0pt}\setlength{\parskip}{0pt}}
\setcounter{secnumdepth}{0}
% Redefines (sub)paragraphs to behave more like sections
\ifx\paragraph\undefined\else
\let\oldparagraph\paragraph
\renewcommand{\paragraph}[1]{\oldparagraph{#1}\mbox{}}
\fi
\ifx\subparagraph\undefined\else
\let\oldsubparagraph\subparagraph
\renewcommand{\subparagraph}[1]{\oldsubparagraph{#1}\mbox{}}
\fi

% set default figure placement to htbp
\makeatletter
\def\fps@figure{htbp}
\makeatother


\date{}

\begin{document}

\section{(4) Limits}\label{limits}

\subsection{Limits of Functions
{[}4.1{]}}\label{limits-of-functions-4.1}

Intuitively limits of functions are the expected value of a function at
points that can't be solved because they are undefined, e.g.\\
\hspace*{0.333em}\\
{\$\frac\{(x-2)left( x+2
\right)\}\{left( x-2
\right)\}\$} would be undefined at x=2, however as x is
made sufficiently close to 2, that value will become arbitrarily close
to 4.

\subsubsection{The Limit Generally}\label{the-limit-generally}

From early calculus the limit of {\emph{f}(\emph{x})}, as {\emph{x}}
approaches {\emph{a}} was said to be some value L, denoted
{lim\textsubscript{\emph{x} → \emph{a}}(\emph{f}(\emph{x})) = \emph{L}}\\[2\baselineskip]{$$\begin\{aligned\}
\forall varepsilon \textgreater{} 0,
\enspace, exists delta:
\& \notag  \&
\qquad 0 \textless{} lvert x-a
\rvert \textless{} delta
\implies lvert fleft( x
\right) - L rvert \textless{}
\varepsilon
\label\{stewartlimdef\}end\{aligned\}$$}\\

\subparagraph{Remarks on this
Definition}\label{remarks-on-this-definition}

Observe that the following statements are equivalent:

\begin{enumerate}
\item
  {\$x\neq c enspace
  \wedge enspace lvert
  x-a \rvert \textless{} delta\$}
\item
  {0 \textless{} \textbar{}\emph{x} − \emph{a}\textbar{}\textless{}\emph{δ}}
\item
  {\textbar{}\emph{x} − \emph{a}\textbar{}∈(0,\emph{δ})}
\end{enumerate}

\subparagraph{Notation}\label{notation}

~\\
\hspace*{0.333em}\\
If {\emph{L}} is a limit of {\emph{f}} at {\emph{c}}, then it is said
that:

\begin{enumerate}
\item
  {\emph{f}} converges to {\emph{L}} at {\emph{c}}
\item
  {\emph{f}(\emph{x})} approaches {\emph{L}} as {\emph{x}} approaches
  {\emph{c}} This is sometimes expressed with the symbolism
  {\emph{f}(\emph{x}) → \emph{L}} as {\emph{x} → \emph{c}}
\end{enumerate}

~\\
And the following notation is used

\begin{enumerate}
\item
  {lim\textsubscript{\emph{x} → \emph{c}}(\emph{f}(\emph{x})) = \emph{L}}
\item
  {lim\textsubscript{\emph{x} → \emph{c}}\emph{f}}
\end{enumerate}

\subsubsection{The Limit Using Cluster
Points}\label{the-limit-using-cluster-points}

In analysis we more or less use the same definition but we introduce the
concept of cluster points to make it more rigorous.

\subparagraph{Neighborhoods {[}2.2.7{]}}\label{neighborhoods-2.2.7}

A neighborhood is an interval about a value, e.g. the
{\emph{ε}}-neighborhood of {\emph{a}} is some set
{\emph{V}\textsubscript{\emph{ε}}(\emph{a})}:\\
{$$\begin\{aligned\}
V\_\{\varepsilon\}(a) = left(
\varepsilon-a, varepsilon+a
\right) \&= left\{ x :
\varepsilon - a \textless{} x \textless{}
\varepsilon + a right\}
\ \&=
\left\{ x : -varepsilon
\textless{} x - a \textless{} \varepsilon
\right\}
\&= \left\{ x: lvert x-a
\rvert \textless{} varepsilon
\right\}
\label\{nebhdef\}end\{aligned\}$$}\\

\subparagraph{Cluster Points}\label{cluster-points}

Let {\emph{c}} be a real number and let {\emph{A}} be a subset of the
real numbers, {\emph{c}} may or may not be contained by {\emph{A}} it
doesn't matter.\\
Take some interval around {\emph{c}}, or rather consider the
{\emph{ε}}-neighborhood of {\emph{c}},\\
if, some value (other than {\emph{c}}) can be found inside that
interval/neighborhood that is also inside {\emph{A}}, regardless of how
small that interval is made, Then {\emph{c}} is said to be a cluster
point of {\emph{A}}.\\
\hspace*{0.333em}\\
i.e., if the following is true\\
{\$\forall varepsilon \textgreater{} 0,
\enspace exists xneq c
\in Acap
V\_\{\varepsilon\}(c)\$}\\
then {\emph{c}} is said to be a cluster point of {\emph{A}}.\\
\hspace*{0.333em}\\
It basically means that there are infinitely infinitesimal points
between any point in {\emph{A}} and the value {\emph{c}}.

Example

\begin{itemize}
\item
  The point {4} of the set {\{3,4,5\}} is not a cluster point of that
  set because a 0.1-neighbourhood of 4 would be the set
  {\emph{V}\textsubscript{0.1}(4)=\{4\}}, this set does not contain a
  value {\emph{x} ≠ 4} that is also inside the original set.\\
\item
  The point 6 of {(1,6) = \{\emph{x}:1\textless{}\emph{x}\textless{}6\}}
  is a cluster point of {(1,6)} because no matter how small a
  neighborhood is made around 6, there will always be values
  {\emph{x} ≠ 6} inside that interval that are also inside {(1,6)} also
  observe that in this case {6 ∉ (1,6)}
\end{itemize}

\subparagraph{Definition of the Limit
{[}4.1.4{]}}\label{definition-of-the-limit-4.1.4}

So this is the definition that we moreso use in this unit and the one to
memorise (or the Neighborhoods one seems simpler to memorise).\\
\hspace*{0.333em}\\
Let {\emph{A} ⊆ ℝ} and let {\emph{c}} be a cluster point of
{\emph{A}}.\\
\hspace*{0.333em}\\
Now take some function:\\
{$$\begin\{aligned\} f: A rightarrow
\mathbb\{R\} label\{clustlimdeffunc\}
\end\{aligned\}$$}\\
It is said that {\emph{L}} is a limit of {\emph{f}} at {\emph{c}} if:\\
{$$\begin\{aligned\} forall
\varepsilon \textgreater{} 0, \&enspace
\exists delta \textgreater{} 0:
\notag
\qquad \& left( x in A
\enspace wedge enspace
0\textless{}\lvert x-c rvert \textless{}
\delta right) implies
\lvert fleft( x right) -
L \rvert \textless{} varepsilon
\tag\{4.1.4\} label\{414\}
\end\{aligned\}$$}\\

What's the Distinction

This is more or less the same as the typical definition given in early
calculus ({[}stewartlimdef{]}), the distinction here is that we have
specified that {\emph{c}} must be a cluster point of {\emph{A}}, this is
more rigorous because c is always such that there are infinitely many
values in any infinitesimal distance between intself and any
{\emph{x} ∈ \emph{A}},\\
So the limit will always mean a continuous approach as we expect, this
is just a more thorough definition.

\subparagraph{Definition using Neigborhoods
{[}4.1.6{]}}\label{definition-using-neigborhoods-4.1.6}

A value {\emph{L}} is said to be the limit of {\emph{f}} as
{\emph{x} → \emph{c}}, denoted
{lim\textsubscript{\emph{x} → \emph{c}}(\emph{f}(\emph{x}))} if and only
if:\\

{2em}{0pt} \emph{For any} given {\emph{ε}}-neighbourhood of {\emph{L}},
{\$\enspace V\_\{varepsilon\}(L)\$}\\
\emph{There exists} a {\emph{δ}}-neighbourhood of {\emph{c}},
{\$\enspace
V\_\{\delta\}left( L
\right)\$}\\

such that: ~\\

{4em}{0pt} =0.7cm \emph{If} {\emph{x} ≠ \emph{c}} is in both {\emph{A}}
and {\emph{V}\textsubscript{\emph{δ}}(\emph{c})}\\
=0.5cm \emph{Then} {\emph{f}(\emph{x})} must be within the neighbourhood
{\emph{V}\textsubscript{\emph{ε}}(\emph{L})}

Formally

{$$\begin\{aligned\} forall
\varepsilon \textgreater{} 0, \& enspace
\exists delta \textgreater{} 0:
\notag  \&
\qquad x neq c, enspace x
\in Acap
V\_\{\varepsilon\}left( L
\right) implies fleft( x
\right) in
V\_\{\delta\}left( c
\right) tag\{4.1.6\}
\label\{416\}end\{aligned\}$$}\\

Defintions ({[}416{]}) and ({[}414{]}) are equivalent, and are both
consistent with the initial less rigorous definition
({[}stewartlimdef{]}).

\subparagraph{Only one Limit Value
{[}4.1.5{]}}\label{only-one-limit-value-4.1.5}

If {\emph{f} : \emph{A} → ℝ} and {\emph{c}} is a cluster point of
{\emph{A}}, then there is only one value L:
{lim\textsubscript{\emph{x} → \emph{c}}(\emph{f}(\emph{x})) = \emph{L}}

\subsubsection{Using Sequences to Define Limits {[} 4.1.8
{]}}\label{using-sequences-to-define-limits-4.1.8}

Now that limits are defined we can use sequences to define them as well,
this will give us more tools to use later and allows a connection to be
made between material of Chapter 3 and 4.

\subparagraph{Definition}\label{definition}

A value {\emph{L}} is said to be the limit of {\emph{f}} as
{\emph{x} → \emph{c}}, denoted
{lim\textsubscript{\emph{x} → \emph{c}}(\emph{f}(\emph{x}))} if and only
if:\\

{2em}{0pt} \emph{For every} sequence
{(\emph{x}\textsubscript{\emph{n}})} in {\emph{A}},\\

{4em}{0pt} \emph{if} {(\emph{x}\textsubscript{\emph{n}})} converges to
{\emph{c}} such that {\emph{x}\textsubscript{\emph{n}} ≠ \emph{c}},\\

{4em}{0pt} \emph{Then} {(\emph{f}(\emph{x}\textsubscript{\emph{n}}))}
converges to {\emph{L}}

~\\
So basically, again, if {\emph{x}} gets close to {\emph{c}},
{\emph{f}(\emph{x})} gets close to L, but we took {\emph{x}} from a
sequence.

\subsubsection{Divergence Criteria {[} 4.1.9
{]}}\label{divergence-criteria-4.1.9}

Now we can use the \emph{Divergence Criteria} from {[}3.4.5{]} to
determine whether or not a limit exists generally or at a point.

\subparagraph{(a) Limit is not a Specific
Value}\label{a-limit-is-not-a-specific-value}

If {\emph{L} ∈ ℝ}, then {\emph{f}} does not have a limit at {\emph{c}},
if and only if:\\
\hspace*{0.333em}\\
There is a sequence {(\emph{x}\textsubscript{\emph{n}})} in {\emph{A}}
with {\emph{x}\textsubscript{\emph{n}} ≠ \emph{c}}, such that:\\
{(\emph{x}\textsubscript{\emph{n}})} converges to {\emph{c}} but the
sequence {\emph{f}(\emph{x}\textsubscript{\emph{n}})} does not converge
to {\emph{L}}

\subparagraph{(b) No Limit whatsover}\label{b-no-limit-whatsover}

If {\emph{L} ∈ ℝ}, then {\emph{f}} does not have a limit at {\emph{c}},
if and only if:\\
\hspace*{0.333em}\\
There is a sequence {(\emph{x}\textsubscript{\emph{n}})} in {\emph{A}}
with {\emph{x}\textsubscript{\emph{n}} ≠ \emph{c}}, such that:\\
{(\emph{x}\textsubscript{\emph{n}})} converges to {\emph{c}} but the
sequence {\emph{f}(\emph{x}\textsubscript{\emph{n}})} does not converge
in {ℝ}

\subparagraph{The Signum Function}\label{the-signum-function}

The Signum function returns the sign of the input value:

{$$\begin\{aligned\} sgnleft( x
\right) \&:= begin\{cases\} +\&1
\qquad text\{for\}
\enspace \$x textgreater
\enspace 0\$  \&0
\qquad text\{for\}
\enspace \$x= 0\$  -\&1
\qquad text\{for\}
\enspace \$x textless
\enspace 0\$ end\{cases\}
\tag\{4.1.10\} label\{4110\}
\ \&=
\frac\{x\}\{lvert x
\rvert\}
\notagend\{aligned\}$$}\\

\subsection{Limit Theorems {[}4.2{]}}\label{limit-theorems-4.2}

These are useful for calculating limits of functions, they are mostly
extensions of {[}3.2{]}.

\subsubsection{Bounded Functions}\label{bounded-functions}

\subparagraph{Definition}\label{definition-1}

Let {\emph{A} ⊆ ℝ}, {\emph{f} : \emph{A} → ℝ} and let {\emph{c} ∈ ℝ} be
a cluster point of {\emph{A}}.\\
It is said that \emph{f is bounded on a neighbourhood of} {\emph{c}}
if:\\

{2em}{0pt} there exists a {\emph{δ}}-neighborhood
{\emph{V}\textsubscript{\emph{δ}}(\emph{c})} and some constant value
{\emph{M} \textgreater{} 0} such that:\\

{4em}{0pt} {\textbar{}\emph{f}(\emph{x})\textbar{}≤\emph{M}} for every
{\emph{x} ∈ \emph{A} ∩ \emph{V}\textsubscript{\emph{δ}}(\emph{c})}

~\\
So basically a function is said to be \emph{bounded on a neighbourhood
of {\emph{c}}} if:

{4em}{0pt} for some interval (It doesn't matter how small) around
{\emph{c}},

{6em}{0pt} {\emph{f}(\emph{x})} can be contained in some interval

~\\

{4em}{0pt} {\$\exists
\delta\textgreater{}0, enspace
\exists M\textgreater{}0:\$}

{6em}{0pt}
{\emph{x} ∈ \emph{V}\textsubscript{\emph{δ}}(\emph{c}) ⟹ \textbar{}\emph{f}(\emph{x})\textbar{}\textless{}\emph{M}}

~\\
So for example:

\begin{itemize}
\item
  {\emph{f}(\emph{x}) = \emph{x}\textsuperscript{3}} is \emph{bounded on
  every neighborhood of every {\emph{x} ∈ ℝ}} whereas,
\item
  {\$g\left( x right) =
  \sfrac\{1\}\{x\}\$} is \emph{\textbf{not} bounded on a
  neighborhood of 0} because {\emph{g}(\emph{x})} tends to infinity as
  {\emph{x} → 0},

  \begin{itemize}
  \item
    {furthermore {\emph{g}(\emph{x})} is \emph{bounded on \textbf{some
    but notall} neighborhoods of 1}, because an interval around 1 must
    not be drawn large enough to encapsulate 0.}
  \end{itemize}
\end{itemize}

\subparagraph{Limits imply Bounded Neighbourhoods {[}4.2.2
{]}}\label{limits-imply-bounded-neighbourhoods-4.2.2}

A function is bounded on a neighborhood of a point that is a limit of
that function.\\
If a function has a limit at {\emph{c}}, then {\emph{f}} must be
\emph{bounded on some neighborhood of {\emph{c}}},\\
this flows from the initial definitions because we know that {\emph{c}}
is a cluster point and that {(\emph{f}(\emph{x}))} moves closer to
{\emph{L}},\\
hence it must be possible to draw a small enough interval (e.g.
horizontal lines on the {\emph{y}}-axis) to contain all
{\emph{f}(\emph{x})} defined by

\subsubsection{Functions and Arithmetic
{[}4.2.3{]}}\label{functions-and-arithmetic-4.2.3}

Just like with sequences we can define arithmetic operations that relate
to addition and multiplication with functions in order to manipulate
them:\\
\hspace*{0.333em}\\
Let {\emph{A} ⊆ ℝ} ,\\
{$$\begin\{aligned\} f: A rightarrow
\mathbb\{R\} qquad g: A
\rightarrow mathbb\{R\}
\qquad h: A rightarrow
\mathbb\{R\}, enspace h(x)
\neq 0, enspace forall x
\in A
\label\{seqdefgen\}end\{aligned\}$$}\\

We define the following Operations {[}4.2, p. 111{]}:

{$$\begin\{aligned\} \{1\} left( f+g
\right) left( x right)
\&:= f\left( x right) +
g\left( x right)
\label\{addfundef\}
\left( f-g right) left( x
\right) \&:= fleft( x
\right) + gleft( x right)
\label\{subfundef\}
\left( fg right) left( x
\right) \&:= fleft( x
\right) times gleft( x
\right) label\{multfundef\}
\ left( bf
\right) left( x right)
\&:= b \times fleft( x
\right) label\{confundef\}
\ left(
\frac\{f\}\{h\}
\right)left( x right)
\&:= \frac\{fleft( x
\right)\}\{hleft( x
\right)\} end\{aligned\}$$}\\

Limits of Function Operations {[}4.2.4{]}

Because the limit of a function is essentially the expected value of the
function around that value, it stands to reason that the limit will
distribute over the basic operations:\\
{Let the functions be defined as they were in ({[}seqdefgen{]}) and let
{\emph{c} ∈ ℝ} be a custer point of {\emph{A}}.}

{$$\begin\{aligned\} lim\_\{x
\rightarrow c\}left( f
\right) = L qquad
\lim\_\{x rightarrow
c\}\left( g right) = M
\quad lim\_\{xrightarrow
c\}\left( h right) = H
\neq 0
\label\{fundeflim\}end\{aligned\}$$}\\

Then the limits are:

{$$\begin\{aligned\} \{2\}
\lim\_\{xrightarrow c\}
\left( f+g right) \&=
\lim\_\{xrightarrow
c\}\left( f right) +
\lim\_\{xrightarrow
c\}\left( g right) \&= L + M
\label\{addlimdef\}
\{[}1em{]}
\lim\_\{xrightarrow c\}
\left( f-g right) \&=
\lim\_\{xrightarrow c\}
\left( f right) -
\lim\_\{xrightarrow c\}
\left( g right) \&= x - y
\label\{minlimdef\}
\{[}1em{]}
\lim\_\{xrightarrow c\}
\left( c cdot f right)
\&= c \cdot
\lim\_\{xrightarrow c\}
\left( f right) \&= c
\cdot x label\{conmultlimdef\}
\{[}1em{]}
\lim\_\{xrightarrow c\}
\left( ftimes g right)
\&= \lim\_\{xrightarrow c\}
\left( f right) times
\lim\_\{xrightarrow c\}
\left( g right) \&= x
\times y label\{multlimdef\}
\{[}1em{]}
\lim\_\{xrightarrow c\}
\left( f/h right) \&=
\lim\_\{xrightarrow c\}
\left( f right) div
\lim\_\{xrightarrow c\}
\left( h right) \&=
\sfrac\{x\}\{y\}
\label\{divlimdef\}end\{aligned\}$$}\\

\subsubsection{Limit Theorems}\label{limit-theorems}

The rest of the chapter just provides values of varios limits.\\
{Let the functions be defined as they were in ({[}seqdefgen{]}) and let
{\emph{c} ∈ ℝ} be a custer point of {\emph{A}}.}

\subparagraph{Limits Captured in Intervals
{[}4.2.6{]}}\label{limits-captured-in-intervals-4.2.6}

~\\
\hspace*{0.333em}\\
\emph{if} {\emph{f}(\emph{x}) ∈ {[}\emph{a},\emph{b}{]}} for all {\$x
\in A, enspace x neq
c\$}, and {lim\textsubscript{\emph{x} → \emph{c}}(\emph{f})} exists,\\

{2em}{0pt} \emph{then} {\emph{f}(\emph{x}) ∈ {[}\emph{a},\emph{b}{]}}

\subparagraph{Squeeze Theorem {[}4.2.7{]}}\label{squeeze-theorem-4.2.7}

if {[}4.2.6{]} is extended to functions, then we have the squeeze
theorem:\\
\emph{if} {\emph{g}} is within an interval defined by the functions
{\emph{f}} and {\emph{h}}:

{$$\begin\{aligned\} fleft( x
\right) leq gleft( x
\right) leq hleft( x
\right), quad forall
x\in A, enspace xneq c
\label\{squeezelimdist\}end\{aligned\}$$}\\

{2em}{0pt} \emph{then} the limit of g must also be 0

{$$\begin\{aligned\}
\lim\_\{xrightarrow
c\}\left( g right)=L
\label\{limisL\}end\{aligned\}$$}\\

\subparagraph{A Positive Limit implies a neighbourhood with Positive
Values}\label{a-positive-limit-implies-a-neighbourhood-with-positive-values}

Let {\emph{A} ⊆ ℝ} and let {\emph{c} ∈ ℝ} be a cluster point of
{\emph{A}} as in (3.4.6) above.\\
\emph{If}:\\
{$$\begin\{aligned\}
\lim\_\{xrightarrow
c\}\left( f right) \textgreater{} 0
\label\{limneighbourpor\}end\{aligned\}$$}\\

\emph{Then:}

{4em}{0pt} there is a neighborhood
{\emph{V}\textsubscript{\emph{δ}}(\emph{c})} such that
{\$f\left( x right) \textgreater{}0,
\enspace forall x in A
\cap V\_\{delta\}left( c
\right)\$}

~\\
This also holds for negative values and basically all it says, in more
rigorous language, is that if the limit point is above the
{\emph{x}}-axis then there's gotta be points to the left and right that
are above the {\emph{x}}-axis as well (because the whole cluster point
thing means everything can be arbitrarily small).\\
\hspace*{0.333em}\\
Although this may start to seem a little pointless, the idea of making
the definitions this rigorous is like writing code in a scripting
language, by using this very precise language, the logical consequences
give us exactly the concept that we want, even though we need to take a
longer or alternate path to get to that concept than we would otherwise
would generally take in order to describe the concept.

\subsection{Extensions of the Limit Concept
{[}4.3{]}}\label{extensions-of-the-limit-concept-4.3}

These are written in a particularly convoluted fashion, however if the
preceeding material is understood the textbook can be used more or less
as a reference, hence these notes will be brief.

\subsubsection{One-Sided Limits
{[}4.3.1{]}}\label{one-sided-limits-4.3.1}

\subparagraph{Definition {[}4.3.1{]}}\label{definition-4.3.1}

Let {\emph{c} ∈ ℝ} be a cluster point of {\$A\cap
\left( c, infty right) =
\left\{ x in A
\enspace : enspace x \textgreater{} c
\right\}\$}\\
It is said that {\emph{L}} is a \emph{Right-hand limit of {\emph{f}} at
{\emph{c}}} and it is written:\\
{$$\begin\{aligned\} lim\_\{x
\rightarrow c\^{}+\} left( f
\right) = L tag\{4.3.1\}
\label\{431\}end\{aligned\}$$}\\

This can be extended to left-hand limits as well.

Definition in Term of Sequences {[}4.3.2{]}

As above it is said that {\emph{L}} is a \emph{Right-hand limit of
{\emph{f}} at {\emph{c}}} if:

{2em}{0pt} Every sequence {(\emph{x}\textsubscript{\emph{n}})} in
{\emph{A}} that converges to {\emph{c}} is such that
{\emph{f}(\emph{x}\textsubscript{\emph{n}})} converges to {\emph{L}},
given that {\$x\_n\textgreater{}c, \enspace
\forall n in
\mathbb\{N\}\$}

\subparagraph{Limit must be equal on both
sides}\label{limit-must-be-equal-on-both-sides}

A limit is defined only if the limit is equal from both directions\\
{$$\begin\{aligned\}
\lim\_\{xrightarrow
c\}\left( f right) = L
\iff lim\_\{xrightarrow
c\^{}+\}\left( f right) = L =
\lim\_\{xrightarrow c
\}\left( f right)
\tag\{3.4.3\}
\label\{343\}end\{aligned\}$$}\\

\subsubsection{Infinite Limits {[}4.3.5{]}}\label{infinite-limits-4.3.5}

Let {\emph{c} ∈ ℝ} be a cluster point of {\emph{A}},\\
It is aid that {\emph{f}} tends to {∞} as {\emph{x} → \emph{c}}, and it
is written:

{$$\begin\{aligned\}
\lim\_\{xrightarrow
c\}\left( f right) =
\infty tag\{4.3.5\}
\label\{435\}end\{aligned\}$$}\\

If {∀\emph{α} ∈ ℝ}, {\$\enspace exists
\delta \textgreater{} 0\$}:

{6em}{0pt}
{0 \textless{} \textbar{}\emph{x} − \emph{c}\textbar{}\textless{}\emph{δ} ⟹ \emph{f}(\emph{x})\textgreater{}\emph{α},  ∀\emph{x} ∈ \emph{A}}

\subparagraph{One-Sided Limits to Infinity
{[}4.3.8{]}}\label{one-sided-limits-to-infinity-4.3.8}

Let {\emph{c} ∈ ℝ} be a cluster point of {\$A\cap
\left( c, infty right) =
\left\{ x in A
\enspace : enspace x \textgreater{} c
\right\}\$},\\
It is aid that {\emph{f}} tends to {∞} as
{\emph{x} → \emph{c}\textsuperscript{+}}, and it is written:

{$$\begin\{aligned\}
\lim\_\{xrightarrow
c\}\left( f right) =
\infty tag\{4.3.8\}
\label\{438\}end\{aligned\}$$}\\

If {∀\emph{α} ∈ ℝ}, {\$\enspace exists
\delta \textgreater{} 0\$}:

{6em}{0pt}
{0 \textless{} \emph{x} − \emph{c} \textless{} \emph{δ} ⟹ \emph{f}(\emph{x})\textgreater{}\emph{α},  ∀\emph{x} ∈ \emph{A}}

\subparagraph{Ordered Functions}\label{ordered-functions}

If {\emph{f}(\emph{x}) \textless{} \emph{g}(\emph{x})}, then:\\
{$$\begin\{aligned\}
\lim\_\{xrightarrow
c\}\left( f right) =
\infty \&implies lim\_\{x
\rightarrow c\}left( g
\right) = infty
\tag\{4.3.7 (a)\}
\label\{437a\}
\lim\_\{xrightarrow
c\}\left( g right) =
-\infty \&implies
\lim\_\{x rightarrow
c\}\left( f right) =
-\infty tag\{4.3.7 (b)\}
\label\{437b\}end\{aligned\}$$}\\

\subsubsection{Limits at Infinity
{[}4.3.10{]}}\label{limits-at-infinity-4.3.10}

It is also useful to talk about limits as {\emph{x}} tends to {∞}

Let {(\emph{a},∞) ⊆ \emph{A} ⊆ ℝ} for some {\emph{ain}ℝ}\\
It is aid that the limit of {\emph{f}} as {\emph{x} → ∞} is {\emph{L}},
and it is written:\\
{$$\begin\{aligned\}
\lim\_\{xrightarrow
\infty\}left( f right) =
L \tag\{4.3.10\}
\label\{4310\}end\{aligned\}$$}\\

If {\$ \forall varepsilon
\textgreater{}0, \enspace exists K
\textgreater{} 0\$}:

{6em}{0pt}
{\emph{x} \textgreater{} \emph{K} ⟹ \textbar{}\emph{f}(\emph{x}) − \emph{L}\textbar{}\textless{}\emph{ε}}

Limits at Infinity in Terms of Sequences {[}4.3.11{]}

equivalently to ({[}4310{]}), the definition can be expressed in terms
of sequences:\\

{2em}{0pt} Every sequence {(\emph{x}\textsubscript{\emph{n}})} in
{\emph{A} ∩ (\emph{a},∞)} that has
{lim(\emph{x}\textsubscript{\emph{n}})=∞} is such that the sequence
{(\emph{f}(\emph{x}\textsubscript{\emph{n}}))} converges to {\emph{L}}

\subparagraph{Infinite Limits at
Infinity}\label{infinite-limits-at-infinity}

So this basically combines {[}4.3.10{]} with {[}4.3.5{]}

Let {(\emph{a},∞) ⊆ \emph{A} ⊆ ℝ} for some {\emph{a} ∈ ℝ}\\
\hspace*{0.333em}\\
It is aid that {\emph{f}} tends to {∞} as {\emph{x} → ∞}, and it is
written:\\
{$$\begin\{aligned\}
\lim\_\{xrightarrow
\infty\}left( f right) =
\infty tag\{4.3.13\}
\label\{4313\}end\{aligned\}$$}\\

If {\$ \forall varepsilon
\textgreater{}0, \enspace exists K
\textgreater{} \alpha\$}:

{6em}{0pt}
{\emph{x} \textgreater{} \emph{K} ⟹ \emph{f}(\emph{x}) \textgreater{} \emph{α}}

Infinite Limits at Infinity in Terms of Sequences {[}4.3.14{]}

equivalently to ({[}4313{]}), the definition can be expressed in terms
of sequences:\\

{2em}{0pt} Every sequence {(\emph{x}\textsubscript{\emph{n}})} in
{\emph{A} ∩ (\emph{a},∞)} that has
{lim(\emph{x}\textsubscript{\emph{n}})=∞} is such that the limit of the
sequence of function values
{lim (\emph{f}(\emph{x}\textsubscript{\emph{n}})) = ∞}

Ratios of Functions

This result uses (4.3.14) to restate (3.6.5) in terms of functions:\\
If {\$g\left( x right) \textgreater{} 0
\enspace forall x \textgreater{} a\$} and
{\emph{L} ≠ 0} is defined:

{$$\begin\{aligned\} lim\_\{xrightarrow
\infty\}left(
\frac\{fleft( x
\right)\}\{gleft( x
\right)\} right)
\tag\{4.3.15\}
\label\{4315\}end\{aligned\}$$}\\

then,

{$$\begin\{aligned\} L \textgreater{} 0
\implies
\lim\_\{xrightarrow
\infty\}left( f right) =
\infty iff lim\_\{x
\rightarrow infty left( g
\right) = infty\}
\tag\{4.3.15 (i)\} label\{4315i\}
\ L \textless{} 0 implies
\lim\_\{xrightarrow
\infty\}left( f right) =
- \infty iff lim\_\{x
\rightarrow infty left( g
\right) = infty\}
\tag\{4.3.15 (ii)\}
\label\{4315ii\}end\{aligned\}$$}\\

\end{document}

\fi
