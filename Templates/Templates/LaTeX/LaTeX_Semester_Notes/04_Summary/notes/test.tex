\hypertarget{limits}{%
\section[\hfill\break
(4)
Limits]{\texorpdfstring{\protect\hypertarget{SECTION00010000000000000000}{}{}\protect\hypertarget{limits}{}{}~\\
(4) Limits}{ (4) Limits}}\label{limits}}

\hypertarget{limits-of-functions-4.1}{%
\subsection[\hfill\break
Limits of Functions
{[}4.1{]}]{\texorpdfstring{\protect\hypertarget{SECTION00011000000000000000}{}{}\protect\hypertarget{limits-of-functions-4.1}{}{}~\\
Limits of Functions
{[}4.1{]}}{ Limits of Functions {[}4.1{]}}}\label{limits-of-functions-4.1}}

Intuitively limits of functions are the expected value of a function at
points that can't be solved because they are undefined, e.g.\\
~\\
\(\frac{(x-2)\left( x+2 \right)}{\left( x-2 \right)}\) would be
undefined at x=2, however as x is made sufficiently close to 2, that
value will become arbitrarily close to 4.

\hypertarget{the-limit-generally}{%
\subsubsection[\hfill\break
The Limit
Generally]{\texorpdfstring{\protect\hypertarget{SECTION00011100000000000000}{}{}\protect\hypertarget{the-limit-generally}{}{}~\\
The Limit Generally}{ The Limit Generally}}\label{the-limit-generally}}

From early calculus the limit of {f}({x}), as {x} approaches {a} was
said to be some value L, denoted
lim\textsubscript{{x} → {a}}({f}({x})) = {L}\\
\[\begin{aligned}
\forall \varepsilon > 0,
\enspace, \exists \delta:
& \notag \\ &
\qquad 0 < \lvert x-a
\rvert < \delta
\implies \lvert f\left( x
\right) - L \rvert <
\varepsilon
\label{stewartlimdef}\end{aligned}\]\\

Remarks on this
Definition\protect\hypertarget{remarks-on-this-definition}{}{}

Observe that the following statements are equivalent:

\begin{enumerate}
\tightlist
\item
  \(x\neq c \enspace  \wedge \enspace \lvert  x-a \rvert < \delta\)
\item
  0 \textless{} \textbar{}{x} − {a}\textbar\textless{}{δ}
\item
  \textbar{}{x} − {a}\textbar∈(0,{δ})
\end{enumerate}

Notation\protect\hypertarget{notation}{}{}

~\\
~\\
If {L} is a limit of {f} at {c}, then it is said that:

\begin{enumerate}
\tightlist
\item
  {f} converges to {L} at {c}
\item
  {f}({x}) approaches {L} as {x} approaches {c} This is sometimes
  expressed with the symbolism {f}({x}) → {L} as {x} → {c}
\end{enumerate}

~\\
And the following notation is used

\begin{enumerate}
\tightlist
\item
  lim\textsubscript{{x} → {c}}({f}({x})) = {L}
\item
  lim\textsubscript{{x} → {c}}{f}
\end{enumerate}

\hypertarget{the-limit-using-cluster-points}{%
\subsubsection[\hfill\break
The Limit Using Cluster
Points]{\texorpdfstring{\protect\hypertarget{SECTION00011200000000000000}{}{}\protect\hypertarget{the-limit-using-cluster-points}{}{}~\\
The Limit Using Cluster
Points}{ The Limit Using Cluster Points}}\label{the-limit-using-cluster-points}}

In analysis we more or less use the same definition but we introduce the
concept of cluster points to make it more rigorous.

Neighborhoods {[}2.2.7{]}\protect\hypertarget{neighborhoods-2.2.7}{}{}

A neighborhood is an interval about a value, e.g. the {ε}-neighborhood
of {a} is some set {V}\textsubscript{{ε}}({a}):\\
\[\begin{aligned}
V_{\varepsilon}(a) = \left(
\varepsilon-a, \varepsilon+a
\right) &= \left\{ x :
\varepsilon - a < x <
\varepsilon + a \right\}
\\ &=
\left\{ x : -\varepsilon
< x - a < \varepsilon
\right\} \\
&= \left\{ x: \lvert x-a
\rvert < \varepsilon
\right\}
\label{nebhdef}\end{aligned}\]\\

Cluster Points\protect\hypertarget{cluster-points}{}{}

Let {c} be a real number and let {A} be a subset of the real numbers,
{c} may or may not be contained by {A} it doesn't matter.\\
Take some interval around {c}, or rather consider the {ε}-neighborhood
of {c},\\
if, some value (other than {c}) can be found inside that
interval/neighborhood that is also inside {A}, regardless of how small
that interval is made, Then {c} is said to be a cluster point of {A}.\\
~\\
i.e., if the following is true\\
\(\forall \varepsilon > 0, \enspace \exists x\neq c \in A\cap V_{\varepsilon}(c)\)\\
then {c} is said to be a cluster point of {A}.\\
~\\
It basically means that there are infinitely infinitesimal points
between any point in {A} and the value {c}.

Example

\begin{itemize}
\tightlist
\item
  The point 4 of the set \{3,4,5\} is not a cluster point of that set
  because a 0.1-neighbourhood of 4 would be the set
  {V}\textsubscript{0.1}(4)=\{4\}, this set does not contain a value
  {x} ≠ 4 that is also inside the original set.\\
\item
  The point 6 of (1,6) = \{{x}:1\textless{}{x}\textless6\} is a cluster
  point of (1,6) because no matter how small a neighborhood is made
  around 6, there will always be values {x} ≠ 6 inside that interval
  that are also inside (1,6) also observe that in this case 6 ∉ (1,6)
\end{itemize}

Definition of the Limit
{[}4.1.4{]}\protect\hypertarget{definition-of-the-limit-4.1.4}{}{}

So this is the definition that we moreso use in this unit and the one to
memorise (or the Neighborhoods one seems simpler to memorise).\\
~\\
Let {A} ⊆ ℝ and let {c} be a cluster point of {A}.\\
~\\
Now take some function:\\
\[\begin{aligned} f: A \rightarrow
\mathbb{R} \label{clustlimdeffunc}
\end{aligned}\]\\
It is said that {L} is a limit of {f} at {c} if:\\
\[\begin{aligned} \forall
\varepsilon > 0, &\enspace
\exists \delta > 0:
\notag \\
\qquad & \left( x \in A
\enspace \wedge \enspace
0<\lvert x-c \rvert <
\delta \right) \implies
\lvert f\left( x \right) -
L \rvert < \varepsilon
\tag{4.1.4} \label{414}
\end{aligned}\]\\

What's the Distinction

This is more or less the same as the typical definition given in early
calculus ({[}stewartlimdef{]}), the distinction here is that we have
specified that {c} must be a cluster point of {A}, this is more rigorous
because c is always such that there are infinitely many values in any
infinitesimal distance between intself and any {x} ∈ {A},\\
So the limit will always mean a continuous approach as we expect, this
is just a more thorough definition.

Definition using Neigborhoods
{[}4.1.6{]}\protect\hypertarget{definition-using-neigborhoods-4.1.6}{}{}

A value {L} is said to be the limit of {f} as {x} → {c}, denoted
lim\textsubscript{{x} → {c}}({f}({x})) if and only if:\\

2em0pt {For any} given {ε}-neighbourhood of {L},
\(\enspace V_{\varepsilon}(L)\)\\
{There exists} a {δ}-neighbourhood of {c},
\(\enspace V_{\delta}\left( L \right)\)\\

such that:~\\

4em0pt =0.7cm {If} {x} ≠ {c} is in both {A} and
{V}\textsubscript{{δ}}({c})\\
=0.5cm {Then} {f}({x}) must be within the neighbourhood
{V}\textsubscript{{ε}}({L})

Formally

\[\begin{aligned} \forall
\varepsilon > 0, & \enspace
\exists \delta > 0:
\notag \\ &
\qquad x \neq c, \enspace x
\in A\cap
V_{\varepsilon}\left( L
\right) \implies f\left( x
\right) \in
V_{\delta}\left( c
\right) \tag{4.1.6}
\label{416}\end{aligned}\]\\

Defintions ({[}416{]}) and ({[}414{]}) are equivalent, and are both
consistent with the initial less rigorous definition
({[}stewartlimdef{]}).

Only one Limit Value
{[}4.1.5{]}\protect\hypertarget{only-one-limit-value-4.1.5}{}{}

If {f} : {A} → ℝ and {c} is a cluster point of {A}, then there is only
one value L: lim\textsubscript{{x} → {c}}({f}({x})) = {L}

\hypertarget{using-sequences-to-define-limits-4.1.8}{%
\subsubsection[\hfill\break
Using Sequences to Define Limits {[} 4.1.8
{]}]{\texorpdfstring{\protect\hypertarget{SECTION00011300000000000000}{}{}\protect\hypertarget{using-sequences-to-define-limits-4.1.8}{}{}~\\
Using Sequences to Define Limits {[} 4.1.8
{]}}{ Using Sequences to Define Limits {[} 4.1.8 {]}}}\label{using-sequences-to-define-limits-4.1.8}}

Now that limits are defined we can use sequences to define them as well,
this will give us more tools to use later and allows a connection to be
made between material of Chapter 3 and 4.

Definition\protect\hypertarget{definition}{}{}

A value {L} is said to be the limit of {f} as {x} → {c}, denoted
lim\textsubscript{{x} → {c}}({f}({x})) if and only if:\\

2em0pt {For every} sequence ({x}\textsubscript{{n}}) in {A},\\

4em0pt {if} ({x}\textsubscript{{n}}) converges to {c} such that
{x}\textsubscript{{n}} ≠ {c},\\

4em0pt {Then} ({f}({x}\textsubscript{{n}})) converges to {L}

~\\
So basically, again, if {x} gets close to {c}, {f}({x}) gets close to L,
but we took {x} from a sequence.

\hypertarget{divergence-criteria-4.1.9}{%
\subsubsection[\hfill\break
Divergence Criteria {[} 4.1.9
{]}]{\texorpdfstring{\protect\hypertarget{SECTION00011400000000000000}{}{}\protect\hypertarget{divergence-criteria-4.1.9}{}{}~\\
Divergence Criteria {[} 4.1.9
{]}}{ Divergence Criteria {[} 4.1.9 {]}}}\label{divergence-criteria-4.1.9}}

Now we can use the {Divergence Criteria} from {[}3.4.5{]} to determine
whether or not a limit exists generally or at a point.

(a) Limit is not a Specific
Value\protect\hypertarget{a-limit-is-not-a-specific-value}{}{}

If {L} ∈ ℝ, then {f} does not have a limit at {c}, if and only if:\\
~\\
There is a sequence ({x}\textsubscript{{n}}) in {A} with
{x}\textsubscript{{n}} ≠ {c}, such that:\\
({x}\textsubscript{{n}}) converges to {c} but the sequence
{f}({x}\textsubscript{{n}}) does not converge to {L}

(b) No Limit whatsover\protect\hypertarget{b-no-limit-whatsover}{}{}

If {L} ∈ ℝ, then {f} does not have a limit at {c}, if and only if:\\
~\\
There is a sequence ({x}\textsubscript{{n}}) in {A} with
{x}\textsubscript{{n}} ≠ {c}, such that:\\
({x}\textsubscript{{n}}) converges to {c} but the sequence
{f}({x}\textsubscript{{n}}) does not converge in ℝ

The Signum Function\protect\hypertarget{the-signum-function}{}{}

The Signum function returns the sign of the input value:

\[\begin{aligned} sgn\left( x
\right) &:= \begin{cases} +&1
\qquad \text{for}
\enspace $x \textgreater
\enspace 0$ \\ &0
\qquad \text{for}
\enspace $x= 0$ \\ -&1
\qquad \text{for}
\enspace $x \textless
\enspace 0$ \end{cases}
\tag{4.1.10} \label{4110}
\\ &=
\frac{x}{\lvert x
\rvert}
\notag\end{aligned}\]\\

\hypertarget{limit-theorems-4.2}{%
\subsection[\hfill\break
Limit Theorems
{[}4.2{]}]{\texorpdfstring{\protect\hypertarget{SECTION00012000000000000000}{}{}\protect\hypertarget{limit-theorems-4.2}{}{}~\\
Limit Theorems
{[}4.2{]}}{ Limit Theorems {[}4.2{]}}}\label{limit-theorems-4.2}}

These are useful for calculating limits of functions, they are mostly
extensions of {[}3.2{]}.

\hypertarget{bounded-functions}{%
\subsubsection[\hfill\break
Bounded
Functions]{\texorpdfstring{\protect\hypertarget{SECTION00012100000000000000}{}{}\protect\hypertarget{bounded-functions}{}{}~\\
Bounded Functions}{ Bounded Functions}}\label{bounded-functions}}

Definition\protect\hypertarget{definition-1}{}{}

Let {A} ⊆ ℝ, {f} : {A} → ℝ and let {c} ∈ ℝ be a cluster point of {A}.\\
It is said that {f is bounded on a neighbourhood of} {c} if:\\

2em0pt there exists a {δ}-neighborhood {V}\textsubscript{{δ}}({c}) and
some constant value {M} \textgreater{} 0 such that:\\

4em0pt \textbar{}{f}({x})\textbar≤{M} for every
{x} ∈ {A} ∩ {V}\textsubscript{{δ}}({c})

~\\
So basically a function is said to be {bounded on a neighbourhood of
{c}} if:

4em0pt for some interval (It doesn't matter how small) around {c},

6em0pt {f}({x}) can be contained in some interval

~\\

4em0pt \(\exists \delta>0, \enspace \exists M>0:\)

6em0pt
{x} ∈ {V}\textsubscript{{δ}}({c}) ⟹ \textbar{}{f}({x})\textbar\textless{}{M}

~\\
So for example:

\begin{itemize}
\tightlist
\item
  {f}({x}) = {x}\textsuperscript{3} is {bounded on every neighborhood of
  every {x} ∈ ℝ} whereas,
\item
  \(g\left( x \right) =  \sfrac{1}{x}\) is {{not} bounded on a
  neighborhood of 0} because {g}({x}) tends to infinity as {x} → 0,

  \begin{itemize}
  \tightlist
  \item
    furthermore {g}({x}) is {bounded on {some but notall} neighborhoods
    of 1}, because an interval around 1 must not be drawn large enough
    to encapsulate 0.
  \end{itemize}
\end{itemize}

Limits imply Bounded Neighbourhoods {[}4.2.2
{]}\protect\hypertarget{limits-imply-bounded-neighbourhoods-4.2.2}{}{}

A function is bounded on a neighborhood of a point that is a limit of
that function.\\
If a function has a limit at {c}, then {f} must be {bounded on some
neighborhood of {c}},\\
this flows from the initial definitions because we know that {c} is a
cluster point and that ({f}({x})) moves closer to {L},\\
hence it must be possible to draw a small enough interval (e.g.
horizontal lines on the {y}-axis) to contain all {f}({x}) defined by

\hypertarget{functions-and-arithmetic-4.2.3}{%
\subsubsection[\hfill\break
Functions and Arithmetic
{[}4.2.3{]}]{\texorpdfstring{\protect\hypertarget{SECTION00012200000000000000}{}{}\protect\hypertarget{functions-and-arithmetic-4.2.3}{}{}~\\
Functions and Arithmetic
{[}4.2.3{]}}{ Functions and Arithmetic {[}4.2.3{]}}}\label{functions-and-arithmetic-4.2.3}}

Just like with sequences we can define arithmetic operations that relate
to addition and multiplication with functions in order to manipulate
them:\\
~\\
Let {A} ⊆ ℝ ,\\
\[\begin{aligned} f: A \rightarrow
\mathbb{R} \qquad g: A
\rightarrow \mathbb{R}
\qquad h: A \rightarrow
\mathbb{R}, \enspace h(x)
\neq 0, \enspace \forall x
\in A
\label{seqdefgen}\end{aligned}\]\\

We define the following Operations {[}4.2, p. 111{]}:

\[\begin{aligned} {1} \left( f+g
\right) \left( x \right)
&:= f\left( x \right) +
g\left( x \right)
\label{addfundef} \\
\left( f-g \right) \left( x
\right) &:= f\left( x
\right) + g\left( x \right)
\label{subfundef} \\
\left( fg \right) \left( x
\right) &:= f\left( x
\right) \times g\left( x
\right) \label{multfundef}
\\ \left( bf
\right) \left( x \right)
&:= b \times f\left( x
\right) \label{confundef}
\\ \left(
\frac{f}{h}
\right)\left( x \right)
&:= \frac{f\left( x
\right)}{h\left( x
\right)} \end{aligned}\]\\

Limits of Function Operations {[}4.2.4{]}

Because the limit of a function is essentially the expected value of the
function around that value, it stands to reason that the limit will
distribute over the basic operations:\\
Let the functions be defined as they were in ({[}seqdefgen{]}) and let
{c} ∈ ℝ be a custer point of {A}.

\[\begin{aligned} \lim_{x
\rightarrow c}\left( f
\right) = L \qquad
\lim_{x \rightarrow
c}\left( g \right) = M
\quad \lim_{x\rightarrow
c}\left( h \right) = H
\neq 0
\label{fundeflim}\end{aligned}\]\\

Then the limits are:

\[\begin{aligned} {2}
\lim_{x\rightarrow c}
\left( f+g \right) &=
\lim_{x\rightarrow
c}\left( f \right) +
\lim_{x\rightarrow
c}\left( g \right) &= L + M
\label{addlimdef}
\\[1em]
\lim_{x\rightarrow c}
\left( f-g \right) &=
\lim_{x\rightarrow c}
\left( f \right) -
\lim_{x\rightarrow c}
\left( g \right) &= x - y
\label{minlimdef}
\\[1em]
\lim_{x\rightarrow c}
\left( c \cdot f \right)
&= c \cdot
\lim_{x\rightarrow c}
\left( f \right) &= c
\cdot x \label{conmultlimdef}
\\[1em]
\lim_{x\rightarrow c}
\left( f\times g \right)
&= \lim_{x\rightarrow c}
\left( f \right) \times
\lim_{x\rightarrow c}
\left( g \right) &= x
\times y \label{multlimdef}
\\[1em]
\lim_{x\rightarrow c}
\left( f/h \right) &=
\lim_{x\rightarrow c}
\left( f \right) \div
\lim_{x\rightarrow c}
\left( h \right) &=
\sfrac{x}{y}
\label{divlimdef}\end{aligned}\]\\

\hypertarget{limit-theorems}{%
\subsubsection[\hfill\break
Limit
Theorems]{\texorpdfstring{\protect\hypertarget{SECTION00012300000000000000}{}{}\protect\hypertarget{limit-theorems}{}{}~\\
Limit Theorems}{ Limit Theorems}}\label{limit-theorems}}

The rest of the chapter just provides values of varios limits.\\
Let the functions be defined as they were in ({[}seqdefgen{]}) and let
{c} ∈ ℝ be a custer point of {A}.

Limits Captured in Intervals
{[}4.2.6{]}\protect\hypertarget{limits-captured-in-intervals-4.2.6}{}{}

~\\
~\\
{if} {f}({x}) ∈ {[}{a},{b}{]} for all \(x \in A, \enspace x \neq c\),
and lim\textsubscript{{x} → {c}}({f}) exists,\\

2em0pt {then} {f}({x}) ∈ {[}{a},{b}{]}

Squeeze Theorem
{[}4.2.7{]}\protect\hypertarget{squeeze-theorem-4.2.7}{}{}

if {[}4.2.6{]} is extended to functions, then we have the squeeze
theorem:\\
{if} {g} is within an interval defined by the functions {f} and {h}:

\[\begin{aligned} f\left( x
\right) \leq g\left( x
\right) \leq h\left( x
\right), \quad \forall
x\in A, \enspace x\neq c
\label{squeezelimdist}\end{aligned}\]\\

2em0pt {then} the limit of g must also be 0

\[\begin{aligned}
\lim_{x\rightarrow
c}\left( g \right)=L
\label{limisL}\end{aligned}\]\\

A Positive Limit implies a neighbourhood with Positive
Values\protect\hypertarget{a-positive-limit-implies-a-neighbourhood-with-positive-values}{}{}

Let {A} ⊆ ℝ and let {c} ∈ ℝ be a cluster point of {A} as in (3.4.6)
above.\\
{If}:\\
\[\begin{aligned}
\lim_{x\rightarrow
c}\left( f \right) > 0
\label{limneighbourpor}\end{aligned}\]\\

{Then:}

4em0pt there is a neighborhood {V}\textsubscript{{δ}}({c}) such that
\(f\left( x \right) >0, \enspace \forall x \in A \cap V_{\delta}\left( c \right)\)

~\\
This also holds for negative values and basically all it says, in more
rigorous language, is that if the limit point is above the {x}-axis then
there's gotta be points to the left and right that are above the
{x}-axis as well (because the whole cluster point thing means everything
can be arbitrarily small).\\
~\\
Although this may start to seem a little pointless, the idea of making
the definitions this rigorous is like writing code in a scripting
language, by using this very precise language, the logical consequences
give us exactly the concept that we want, even though we need to take a
longer or alternate path to get to that concept than we would otherwise
would generally take in order to describe the concept.

\hypertarget{extensions-of-the-limit-concept-4.3}{%
\subsection[\hfill\break
Extensions of the Limit Concept
{[}4.3{]}]{\texorpdfstring{\protect\hypertarget{SECTION00013000000000000000}{}{}\protect\hypertarget{extensions-of-the-limit-concept-4.3}{}{}~\\
Extensions of the Limit Concept
{[}4.3{]}}{ Extensions of the Limit Concept {[}4.3{]}}}\label{extensions-of-the-limit-concept-4.3}}

These are written in a particularly convoluted fashion, however if the
preceeding material is understood the textbook can be used more or less
as a reference, hence these notes will be brief.

\hypertarget{one-sided-limits-4.3.1}{%
\subsubsection[\hfill\break
One-Sided Limits
{[}4.3.1{]}]{\texorpdfstring{\protect\hypertarget{SECTION00013100000000000000}{}{}\protect\hypertarget{one-sided-limits-4.3.1}{}{}~\\
One-Sided Limits
{[}4.3.1{]}}{ One-Sided Limits {[}4.3.1{]}}}\label{one-sided-limits-4.3.1}}

Definition {[}4.3.1{]}\protect\hypertarget{definition-4.3.1}{}{}

Let {c} ∈ ℝ be a cluster point of
\(A\cap \left( c, \infty \right) = \left\{ x \in A \enspace : \enspace x > c \right\}\)\\
It is said that {L} is a {Right-hand limit of {f} at {c}} and it is
written:\\
\[\begin{aligned} \lim_{x
\rightarrow cˆ+} \left( f
\right) = L \tag{4.3.1}
\label{431}\end{aligned}\]\\

This can be extended to left-hand limits as well.

Definition in Term of Sequences {[}4.3.2{]}

As above it is said that {L} is a {Right-hand limit of {f} at {c}} if:

2em0pt Every sequence ({x}\textsubscript{{n}}) in {A} that converges to
{c} is such that {f}({x}\textsubscript{{n}}) converges to {L}, given
that \(x_n>c, \enspace \forall n \in \mathbb{N}\)

Limit must be equal on both
sides\protect\hypertarget{limit-must-be-equal-on-both-sides}{}{}

A limit is defined only if the limit is equal from both directions\\
\[\begin{aligned}
\lim_{x\rightarrow
c}\left( f \right) = L
\iff \lim_{x\rightarrow
cˆ+}\left( f \right) = L =
\lim_{x\rightarrow c
}\left( f \right)
\tag{3.4.3}
\label{343}\end{aligned}\]\\

\hypertarget{infinite-limits-4.3.5}{%
\subsubsection[\hfill\break
Infinite Limits
{[}4.3.5{]}]{\texorpdfstring{\protect\hypertarget{SECTION00013200000000000000}{}{}\protect\hypertarget{infinite-limits-4.3.5}{}{}~\\
Infinite Limits
{[}4.3.5{]}}{ Infinite Limits {[}4.3.5{]}}}\label{infinite-limits-4.3.5}}

Let {c} ∈ ℝ be a cluster point of {A},\\
It is aid that {f} tends to ∞ as {x} → {c}, and it is written:

\[\begin{aligned}
\lim_{x\rightarrow
c}\left( f \right) =
\infty \tag{4.3.5}
\label{435}\end{aligned}\]\\

If ∀{α} ∈ ℝ, \(\enspace \exists \delta > 0\):

6em0pt
0 \textless{} \textbar{}{x} − {c}\textbar\textless{}{δ} ⟹ {f}({x})\textgreater{}{α},  ∀{x} ∈ {A}

One-Sided Limits to Infinity
{[}4.3.8{]}\protect\hypertarget{one-sided-limits-to-infinity-4.3.8}{}{}

Let {c} ∈ ℝ be a cluster point of
\(A\cap \left( c, \infty \right) = \left\{ x \in A \enspace : \enspace x > c \right\}\),\\
It is aid that {f} tends to ∞ as {x} → {c}\textsuperscript{+}, and it is
written:

\[\begin{aligned}
\lim_{x\rightarrow
c}\left( f \right) =
\infty \tag{4.3.8}
\label{438}\end{aligned}\]\\

If ∀{α} ∈ ℝ, \(\enspace \exists \delta > 0\):

6em0pt
0 \textless{} {x} − {c} \textless{} {δ} ⟹ {f}({x})\textgreater{}{α},  ∀{x} ∈ {A}

Ordered Functions\protect\hypertarget{ordered-functions}{}{}

If {f}({x}) \textless{} {g}({x}), then:\\
\[\begin{aligned}
\lim_{x\rightarrow
c}\left( f \right) =
\infty &\implies \lim_{x
\rightarrow c}\left( g
\right) = \infty
\tag{4.3.7 (a)}
\label{437a}\\
\lim_{x\rightarrow
c}\left( g \right) =
-\infty &\implies
\lim_{x \rightarrow
c}\left( f \right) =
-\infty \tag{4.3.7 (b)}
\label{437b}\end{aligned}\]\\

\hypertarget{limits-at-infinity-4.3.10}{%
\subsubsection[\hfill\break
Limits at Infinity
{[}4.3.10{]}]{\texorpdfstring{\protect\hypertarget{SECTION00013300000000000000}{}{}\protect\hypertarget{limits-at-infinity-4.3.10}{}{}~\\
Limits at Infinity
{[}4.3.10{]}}{ Limits at Infinity {[}4.3.10{]}}}\label{limits-at-infinity-4.3.10}}

It is also useful to talk about limits as {x} tends to ∞

Let ({a},∞) ⊆ {A} ⊆ ℝ for some {ain}ℝ\\
It is aid that the limit of {f} as {x} → ∞ is {L}, and it is written:\\
\[\begin{aligned}
\lim_{x\rightarrow
\infty}\left( f \right) =
L \tag{4.3.10}
\label{4310}\end{aligned}\]\\

If \$ \textbackslash forall \textbackslash varepsilon \textgreater0,
\textbackslash enspace \textbackslash exists K \textgreater{} 0\$:

6em0pt
{x} \textgreater{} {K} ⟹ \textbar{}{f}({x}) − {L}\textbar\textless{}{ε}

Limits at Infinity in Terms of Sequences {[}4.3.11{]}

equivalently to ({[}4310{]}), the definition can be expressed in terms
of sequences:\\

2em0pt Every sequence ({x}\textsubscript{{n}}) in {A} ∩ ({a},∞) that has
lim({x}\textsubscript{{n}})=∞ is such that the sequence
({f}({x}\textsubscript{{n}})) converges to {L}

Infinite Limits at
Infinity\protect\hypertarget{infinite-limits-at-infinity}{}{}

So this basically combines {[}4.3.10{]} with {[}4.3.5{]}

Let ({a},∞) ⊆ {A} ⊆ ℝ for some {a} ∈ ℝ\\
~\\
It is aid that {f} tends to ∞ as {x} → ∞, and it is written:\\
\[\begin{aligned}
\lim_{x\rightarrow
\infty}\left( f \right) =
\infty \tag{4.3.13}
\label{4313}\end{aligned}\]\\

If \$ \textbackslash forall \textbackslash varepsilon \textgreater0,
\textbackslash enspace \textbackslash exists K \textgreater{}
\textbackslash alpha\$:

6em0pt {x} \textgreater{} {K} ⟹ {f}({x}) \textgreater{} {α}

Infinite Limits at Infinity in Terms of Sequences {[}4.3.14{]}

equivalently to ({[}4313{]}), the definition can be expressed in terms
of sequences:\\

2em0pt Every sequence ({x}\textsubscript{{n}}) in {A} ∩ ({a},∞) that has
lim({x}\textsubscript{{n}})=∞ is such that the limit of the sequence of
function values lim ({f}({x}\textsubscript{{n}})) = ∞

Ratios of Functions

This result uses (4.3.14) to restate (3.6.5) in terms of functions:\\
If \(g\left( x \right) > 0 \enspace \forall x > a\) and {L} ≠ 0 is
defined:

\[\begin{aligned} lim_{x\rightarrow
\infty}\left(
\frac{f\left( x
\right)}{g\left( x
\right)} \right)
\tag{4.3.15}
\label{4315}\end{aligned}\]\\

then,

\[\begin{aligned} L > 0
\implies
\lim_{x\rightarrow
\infty}\left( f \right) =
\infty \iff \lim_{x
\rightarrow \infty \left( g
\right) = \infty}
\tag{4.3.15 (i)} \label{4315i}
\\ L < 0 \implies
\lim_{x\rightarrow
\infty}\left( f \right) =
- \infty \iff \lim_{x
\rightarrow \infty \left( g
\right) = \infty}
\tag{4.3.15 (ii)}
\label{4315ii}\end{aligned}\]\\

\hfill\break

\begin{center}\rule{0.5\linewidth}{0.5pt}\end{center}
